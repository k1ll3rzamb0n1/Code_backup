\documentclass[12pt]{article}
\usepackage{amsmath}
\usepackage{graphicx}
\usepackage{pdfpages}
\usepackage{amssymb}

\title{Exercise Set 6}
\author{Ryan C. Bleile}

\begin{document}
\maketitle

\section{•}
Show S = span ($v_1 , v_2 , ... , v_n$) is a n dimensional subspace of X.\\
If X is an M-dimensional subspace then $v_1 \ \in$ X, $v_2 \ \in$ X, and $v_n \ \in$ X. So all vectors $v_1 \rightarrow v_n$ are in X. The vector space X can be identified as a linear combination of its basis vectors. One common basis to choose which we can identify all vector is X with is its standard basis. Therefore:

\[
X \equiv
x_1
\begin{bmatrix}
1\\
0\\
...\\
0
\end{bmatrix}
+ x_2
\begin{bmatrix}
0\\
1\\
...\\
0
\end{bmatrix}
+ ...
+ x_m
\begin{bmatrix}
0\\
...\\
0\\
1
\end{bmatrix}
\]

All vectors that lie in the vector space X can be rewritten as a particular linear combination of the basis of X. So vector $v_1$ can be rewritten as: 


\[
v_1 \equiv
C_1
\begin{bmatrix}
1\\
0\\
...\\
0
\end{bmatrix}
+ C_2
\begin{bmatrix}
0\\
1\\
...\\
0
\end{bmatrix}
+ ...
+ C_m
\begin{bmatrix}
0\\
...\\
0\\
1
\end{bmatrix}
\]

Also, two vectors added or scaled in any way could be rewritten in this form as another vector of the vector space, X. Such that:


\[
v_1 + v_2 \equiv
(C_{1_1} + C_{1_2} )
\begin{bmatrix}
1\\
0\\
...\\
0
\end{bmatrix}
+ (C_{2_1} + C_{2_2} )
\begin{bmatrix}
0\\
1\\
...\\
0
\end{bmatrix}
+ ...
+ (C_{m_1} + C_{m_2} )
\begin{bmatrix}
0\\
...\\
0\\
1
\end{bmatrix}
\]
And since $(C_{1_1} + C_{1_2} ) \in \mathbb{R}$ we can rewrite this as another constant say $C_1$. There for any linear combination of vectors in the vector space can be added and scaled to be rewritten as a vector lying in the space using its standard basis. Thus any linear combination of vectors in X produces vectors in X, and since our definition of a vector space is a linear combination of vectors in the space will produce another vector in the space we can see that for any subspace of X such that S = span($v_1, v_2, ..., v_n$) will be a subspace of X. Since a span in the linear combination of vectors and the basis of all the vectors in S can be written using the same standard basis as X.\\

If the dimension of X is exactly equal to the number of vectors in S than we know that S is one possible basis for X as long as no vectors in S are scalar multiples of each other. Also, If the dimension of X is greater than S than we know that vectors in S are not linearly independent and therefore cannot be a basis of X. If the number of vectors in S are less than the dimensions of X than we know S is a subspace of X and is not a basis for X.

\section{•}

We may write a vector in X using the basis $v_1,\ v_2 ,\ v_3,\ ...,\ v_n$ and we call that vector v. If we wish too write v as:
$$ v = a_1 v_1 + a_2 v_2 + ... + a_n v_n  $$
But we also wish to write v as:
$$ v = b_1 v_1 + b_2 v_2 + ... + b_n v_n $$

than we are actually writing the same vector. Since $a_i \in \mathbb{R}$ and $b_i \in \mathbb{R}$ and the basis vectors used for both are the same we are essentially writing the same vector twice and all of $a_i$ will equal $b_i$ for every $i$.\\

This can be shown by first looking at the vector v. We know that v must equal v and v-v = 0. Using this knowledge we can write out the vector v in both of its forms.
\begin{eqnarray*}
v &=& v\\
a_1 v_1 + a_2 v_2 + ... + a_n v_n &=& b_1 v_1 + b_2 v_2 + ... + b_n v_n\\
(a_1 - b_1) v_1 + (a_2 - b_2) v_2 + ... (a_n - b_n) v_n &=& 0;\\
\therefore a_1 - b_1\ AND\ a_2 - b_2\ AND\ all\ a_n - b_n  &=& 0
\end{eqnarray*} 
$$\therefore a_1 = b_1\ AND\ a_2 = b_2\ AND\ a_n = b_n $$
And we have shown that $a_i = b_i$ for all i when these vectors are compared in the same basis.

\section{•}
 
If we take $P^2$ which is the space of all quadratic polynomials we can represent it in standard basis as:

\[P^2 \equiv a_0 + a_1 x + a_2 x^2 \equiv
a_0
\begin{bmatrix}
1\\
0\\
0
\end{bmatrix}
+ a_1
\begin{bmatrix}
0\\
1\\
0
\end{bmatrix}
+ a_2
\begin{bmatrix}
0\\
0\\
1
\end{bmatrix}
\]

where the standard basis of $P^2$ is:

\[ basis(P^2) = (1, x, x^2) 
\equiv \begin{bmatrix}
1\\
0\\
0
\end{bmatrix}
,
\begin{bmatrix}
0\\
x\\
0
\end{bmatrix}
,
\begin{bmatrix}
0\\
0\\
x^2
\end{bmatrix}
\]
We can choose to write out $P^2$ in terms of a different basis of our choice. One possible choice would be to pick a non-standard basis such as:

$$basis = (1-x, x^2 + x, 1 + x + x^2) $$
or even:
$$ basis = (1- x - x^2, x -1 - x^2, 1 + x^2) $$

These are but two out of an infinite possibility of ways to describe the space made by $P^2$. If we look to the standard basis where $P^2$ is identified with vectors of only numbers we can represent the space of $P^2$ with an infinite number of 3, 3x1 vectors which are linearly independent, since $P^2$ is isomorphic to the vector space $\mathbb{R}^3$.

\section{•}

Given a linear operator L:$\mathbb{R}^2 \rightarrow \mathbb{R}^2$ which can be represented in the standard basis as:
\[
mat(L) = 
\begin{bmatrix}
1 & 4\\
2 & 3
\end{bmatrix}
\]

If we wish, we are able to rewrite L in terms of a non-standard basis $v_1, v_2$ where:
\[
v_1 = 
\begin{bmatrix}
1\\
1
\end{bmatrix}
\ AND\
v_2 = 
\begin{bmatrix}
-2\\
1
\end{bmatrix}
\]

We start by writing taking the basis of L and finding a way to rewrite then in terms of the new basis $v_1, v_2$. If we start with the basis of L to be standard than they are:
\[
\begin{bmatrix}
1\\
0
\end{bmatrix}
\ AND\ 
\begin{bmatrix}
0\\
1
\end{bmatrix}
\]
If we write $v_1$ and $v_2$ in terms of these basis we will have a way to convert between the two. So:

\[
\begin{bmatrix}
1\\
0
\end{bmatrix}
=
C_1
\begin{bmatrix}
1\\
1
\end{bmatrix}
+
C_2
\begin{bmatrix}
-2\\
1
\end{bmatrix}
\]
If we now expand the v basis out in terms of the standard basis we see that:
\[
\begin{bmatrix}
1\\
0
\end{bmatrix}
=
C_1
\begin{bmatrix}
1\\
0
\end{bmatrix}
+C_1
\begin{bmatrix}
0\\
1
\end{bmatrix}
+
C_2
\begin{bmatrix}
-2\\
0
\end{bmatrix}
+
C_2
\begin{bmatrix}
0\\
1
\end{bmatrix}
\]
Now combining like terms we see:
$$ 1 = C_1 - 2 C_2 $$
AND
$$ 0 = C_1 + C_2 $$

We can now solve this system of equations to find that: $C_1$ = $\frac{1}{3}$ and $C_2$ = $\frac{-1}{3}$. Doing this same process again for the other basis vector we get to this system of equations:
$$ 0 = C_3 - 2 C_2$$
AND
$$ 1 = C_3 + C_4 $$

Which give us solutions for $C_3$ and $C_4$ to be: $C_3$ = $\frac{2}{3}$ and $C_4$ = $\frac{1}{3}$. From this we need only put together out coefficients in terms of a matrix to see that the linear operator L written in terms of the v bases described above is: 
\[
mat(L) = 
\begin{bmatrix}
\frac{1}{3} & \frac{-1}{3}\\
\frac{2}{3} & \frac{1}{3}
\end{bmatrix}
\]

Where now each component of the matrix is relative to the new basis $v_1$ and $v_2$

\section{•}
The trace of a 2x2 matrix is such that a matrix:
\[
mat() = 
\begin{bmatrix}
a & b\\
c & d
\end{bmatrix}
\]

is the Sum of the diagonals a and d: a+d. A traceless matrix is one than where a+d = 0. which could also be defined as a = -d. So a traceless matrix would be any matrix that looked like:
\[
mat() = 
\begin{bmatrix}
a & b\\
c & -a
\end{bmatrix}
\]

If X is the set of all traceless matrices than X is a vector space. We know X is a vector space because we can write it as the span of its basis. X's basis are:
\[
basis = 
\begin{bmatrix}
1& 0\\
0 & -1
\end{bmatrix}
,
\begin{bmatrix}
0 & 1\\
0 & 0
\end{bmatrix}
,
\begin{bmatrix}
0 & 0\\
1 & 0
\end{bmatrix}
\]

such that the span of the basis will produce the vector space X:

\[
X = 
span(
\begin{bmatrix}
1& 0\\
0 & -1
\end{bmatrix}
,
\begin{bmatrix}
0 & 1\\
0 & 0
\end{bmatrix}
,
\begin{bmatrix}
0 & 0\\
1 & 0
\end{bmatrix}
)
\]

meaning that:
\[
X = 
a
\begin{bmatrix}
1& 0\\
0 & -1
\end{bmatrix}
+ b
\begin{bmatrix}
0 & 1\\
0 & 0
\end{bmatrix}
+ c
\begin{bmatrix}
0 & 0\\
1 & 0
\end{bmatrix}
\]

Just to check this result with our definition of a vector space we can see that X will have a zero vector on the all zero input. Also, We can write a linear combination of Vectors in X as another vector in X such that:
\[
C_1
\begin{bmatrix}
1& 0\\
0 & -1
\end{bmatrix}
+ C_2
\begin{bmatrix}
0 & 1\\
0 & 0
\end{bmatrix}
+ C_3
\begin{bmatrix}
0 & 0\\
1 & 0
\end{bmatrix}
+
C_4
\begin{bmatrix}
1& 0\\
0 & -1
\end{bmatrix}
+ C_5
\begin{bmatrix}
0 & 1\\
0 & 0
\end{bmatrix}
+ C_6
\begin{bmatrix}
0 & 0\\
1 & 0
\end{bmatrix}
=
\]
\[
(C_1 + C_4)
\begin{bmatrix}
1& 0\\
0 & -1
\end{bmatrix}
+ (C_2 + C_5)
\begin{bmatrix}
0 & 1\\
0 & 0
\end{bmatrix}
+ (C_3 + C_6)
\begin{bmatrix}
0 & 0\\
1 & 0
\end{bmatrix}
\]
Which since $C_n \in \mathbb{R}$ and combination of C's will be $\in \mathbb{R}$. So this linear combination of vectors in X is:

\[
a
\begin{bmatrix}
1& 0\\
0 & -1
\end{bmatrix}
+ b
\begin{bmatrix}
0 & 1\\
0 & 0
\end{bmatrix}
+ c
\begin{bmatrix}
0 & 0\\
1 & 0
\end{bmatrix}
\]
$$a, b, c \in \mathbb{R} $$

Thus X is a 3 dimensional vector space


\end{document}}