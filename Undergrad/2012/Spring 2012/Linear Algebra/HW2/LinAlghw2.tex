\documentclass[12pt]{article}
\usepackage{amsmath}
\usepackage{graphicx}
\usepackage{pdfpages}

\title{Exercise Set 2}
\author{Ryan C. Bleile}

\begin{document}
\maketitle

\section*{Operators}

There are three operators that will will consider.

\begin{enumerate}
\item	$T_{1}$ : f $\rightarrow$ $x \frac{df}{dx}$ The domain of the operator is quadratic polynomials:\\
		$f = a_{0} + a_{1}x + a_{2}x^{2}$
\item	$T_{2}$ takes points in the plane. The action is rotation by angle $\theta$ about the origin.
\item	$T_{3}$ takes as its input the position, velocity, and acceleration of a free falling object; the output is the position, velocity, and acceleration 1 second later. (Note that the equation of motion of the object can be stated as $\frac{d^{3}x}{dt^{3}} = 0$ where x(t) is the position.) 
\end{enumerate}

\section{Problem 1}

Represent each operator with a matrix.\\

\subsection{$T_{1}$}

In order to write $T_{1}$ in matrix form, first we will define our function as a vector, secondly preform the operation once on the function, and thirdly we will write the solution as a vector using the same definition from the function space.\\
\\
Starting with the function space:
\[
f = a_{0} + a_{1}x + a_{2}x^{2} \equiv
\begin{bmatrix}
a_{0}\\
a_{1}\\
a_{2}
\end{bmatrix}
\]

Now we must preform the operation $T_{1}$ on f\\

\begin{eqnarray*}
T[f] &=& x( \frac{d}{dx}(a_{0}) + \frac{d}{dx}(a_{1}x) + \frac{d}{dx}(a_{2}x^{2}) )\\
&=&		0 + a_{1}x + 2a_{2}x^{2}
\end{eqnarray*}
\[
T[f] = 
\begin{bmatrix}
0\\
a_{1}\\
2a_{2}
\end{bmatrix}
\]

Now working backward we can form the matrix that would perform this operation.

\[
\begin{bmatrix}
 & & \\
 &T& \\
 & & 
\end{bmatrix}
\begin{bmatrix}
a_{0}\\
a_{1}\\
a_{2}
\end{bmatrix}
=
\begin{bmatrix}
0\\
a_{1}\\
2a_{2}
\end{bmatrix}
\]

filling in the values required to turn the initial vector into the final vector we find that the operator $T_{1}$ is

\[
T_{1} = 
\begin{bmatrix}
0 & 0 & 0\\
0 & 1 & 0\\
0 & 0 & 2
\end{bmatrix}
\]

\subsection{$T_{2}$}

In order to derive a rotation matrix for the coordinates (x,y) we will start first with a rotation of zero degrees. In order to get this result from an x, y vector we can write:

\[
\begin{bmatrix}
x\\
y
\end{bmatrix}
=
x
\begin{bmatrix}
1\\
0
\end{bmatrix}
+
y
\begin{bmatrix}
0\\
1
\end{bmatrix}
\]

This is the case when the rotation ($\theta$) = 0. In order to write a formula that emulates this we may write the zero's as $\sin \theta$ and the one's as $\cos \theta$. However in order to achieve the proper rotation, the $\sin$ term that is on the y component must be multiplied by -1. This was derived through trial and error process. This means that in order to transform the vector:
\[
\begin{bmatrix}
x\\
y
\end{bmatrix}
\]

into a rotated vector of the form:

\[
\begin{bmatrix}
x'\\
y'
\end{bmatrix}
\]

we would need a matrix of the form:

\[
\begin{bmatrix}
\cos \theta & - \sin \theta\\
\sin \theta & \cos \theta
\end{bmatrix}
\]

\subsection{$T_{3}$}
Operator $T_{3}$ is an action upon the vector:\\
\[
\begin{bmatrix}
x_{0}\\
v_{0}\\
a_{0}
\end{bmatrix}
\]

Moving from the definition 
$$ \frac{d^{3}x}{dt^{3}} = 0 $$
we will integrate in order to find expressions for a, v, and x. From our basic physics knowledge it is known that
\begin{eqnarray*}
\frac{d^{2}x}{dt^{2}} = a\\
\frac{dx}{dt} = v
\end{eqnarray*}

Also, with the knowledge that a is constant we are able to derive a value for v and then a value for x. To find v will will integrate a.
\begin{eqnarray*}
v &=& \int a dt\\
&=& at + c \\
v(0) &=& v_{0}\\
v(t) &=& at + v_{0}
\end{eqnarray*}

Now we can integrate our solution for v in order to find x:
\begin{eqnarray*}
x &=& \int v(t) dt\\
&=& \int at + v_{0}\\
&=& \frac{1}{2}at^{2} + v_{0}t + c\\
x(0) &=& x_{0}\\
x(t) &=& \frac{1}{2}at^{2} + v_{0}t + x_{0} 
\end{eqnarray*}

now we have an equation representing a, v and x. 

If we apply the transformation now to a,v,x than we will be able find a matrix representing $T_{3}$. Transformation $T_{3}$ applied to the vector 
\[
\begin{bmatrix}
x_{0}\\
v_{0}\\
a_{0}
\end{bmatrix}
\]

applying the operation one step at a time:\\
$a_{0}$ is transformed into $a_{0}$\\
$v_{0}$ is transformed into $a_{0} + v_{0}$\\
$x_{0}$ is transformed into $\frac{1}{2}a_{0} + v_{0} + x_{0}$\\
giving us a vector:\\
\[
\begin{bmatrix}
x_{0} + v_{0} + \frac{1}{2}a_{0}\\
v_{0} + a_{0}\\
a_{0}
\end{bmatrix}
\]

Using the same process to work backwards and find the matrix we see that:\\

\[
\begin{bmatrix}
& &\\
&T&\\
&&
\end{bmatrix}
*
\begin{bmatrix}
x_{0}\\
v_{0}\\
a_{0}
\end{bmatrix}
=
\begin{bmatrix}
x_{0} + v_{0} + \frac{1}{2}a_{0}\\
v_{0} + a_{0}\\
a_{0}
\end{bmatrix}
\]
and this gives us
\[
\begin{bmatrix}
1 & 1 & \frac{1}{2}\\
0 & 1 & 1\\
0 & 0 & 1
\end{bmatrix}
*
\begin{bmatrix}
x_{0}\\
v_{0}\\
a_{0}
\end{bmatrix}
=
\begin{bmatrix}
x_{0} + v_{0} + \frac{1}{2}a_{0}\\
v_{0} + a_{0}\\
a_{0}
\end{bmatrix}
\]


\section{Problem 2}

Using Matrix representation find compute $T_{1}^{2}(x^{2})$\\
\\
Now if we write out the operator in matrix form and multiply it by itself then we can find an operator for $T_{1}^{2}$\\

\[
\begin{bmatrix}
0 & 0 & 0\\
0 & 1 & 0\\
0 & 0 & 2
\end{bmatrix}
*
\begin{bmatrix}
0 & 0 & 0\\
0 & 1 & 0\\
0 & 0 & 2
\end{bmatrix}
=
\begin{bmatrix}
0 & 0 & 0\\
0 & 1 & 0\\
0 & 0 & 4
\end{bmatrix}
\]

Applying the $T_{1}^{2}$ on $x^{2}$, first we must write $x^{2}$ as a vector. Using the same form as we defined for this domain.\\

\[
x^{2} \equiv 
\begin{bmatrix}
0\\
0\\
1
\end{bmatrix}
\]

and now applying the $T^{2}_{1}$ operator:

\[
\begin{bmatrix}
0 & 0 & 0\\
0 & 1 & 0\\
0 & 0 & 4
\end{bmatrix}
*
\begin{bmatrix}
0\\
0\\
1
\end{bmatrix}
=
\begin{bmatrix}
0\\
0\\
4
\end{bmatrix}
\]

\section{Problem 3}

An inverse of the rotation matrix can be found through a trial and error process or geometric reasoning. In order for the coordinate (x,y)to be rotated clockwise instead of counter clock wise than the sine term on the y coordinate must be positive and the sine term on the x coordinate must be negative. This produces this matrix:
\[
T_{2}^{-1} =
\begin{bmatrix}
\cos \theta & \sin \theta\\
- \sin \theta & \cos \theta
\end{bmatrix}
\]

When the operator $T_{3}$ is multiplied with $T_{3}^{-1}$ than it should produce a net effect of no rotation. As should the other multiplication $T_{3}^{-1}$ times $T_{3}$. We can show this by doing the matrix multiplication as seeing the resultant matrix:\\
\\
$T_{3}^{-1}$*$T_{3}$
\[
\begin{bmatrix}
\cos \theta & \sin \theta\\
- \sin \theta & \cos \theta
\end{bmatrix}
*
\begin{bmatrix}
\cos \theta & -\sin \theta\\
\sin \theta & \cos \theta
\end{bmatrix}
=
\begin{bmatrix}
1 & 0\\
0 & 1
\end{bmatrix}
\]
$T_{3}$*$T_{3}^{-1}$
\[
\begin{bmatrix}
\cos \theta & -\sin \theta\\
\sin \theta & \cos \theta
\end{bmatrix}
*
\begin{bmatrix}
\cos \theta & \sin \theta\\
- \sin \theta & \cos \theta
\end{bmatrix}
=
\begin{bmatrix}
1 & 0\\
0 & 1
\end{bmatrix}
\]



\section{Problem 4}

Computing $T^{n}_{3}$\\
\[
T_{3} =
\begin{bmatrix}
1 & 1 & .5\\
0 & 1 & 1\\
0 & 0 & 1
\end{bmatrix}
\]
\[
T^{2}_{3} =
\begin{bmatrix}
1 & 2 & 2\\
0 & 1 & 2\\
0 & 0 & 1
\end{bmatrix}
\]
\[
T^{3}_{3} =
\begin{bmatrix}
1 & 3 & 4.5\\
0 & 1 & 3\\
0 & 0 & 1
\end{bmatrix}
\]
\[
T^{4}_{3} = 
\begin{bmatrix}
1 & 4 & 8\\
0 & 1 & 4\\
0 & 0 & 1
\end{bmatrix}
\]

From this we see that the general formula for $T^{n}_{3}$ is:
\[
T^{n}_{3} = 
\begin{bmatrix}
1 & n & \frac{n^{2}}{2}\\
0 & 1 & n\\
0 & 0 & 1
\end{bmatrix}
\]

The physical meaning of the $T^{2}_{3}$ operator is that instead of giving the acceleration, velocity, and displacement one second ahead of the given vector it will give it two seconds a head (t+2 instead of t+1).

\end{document}